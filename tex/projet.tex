\documentclass[12pt]{article}

\usepackage{minted}
\usepackage[utf8]{inputenc}
\usepackage{url}
\usepackage{latexsym,amsfonts,amssymb,amsthm,amsmath}
\usepackage{mathtools}
\usepackage{enumerate}  

\setlength{\parindent}{0in}
\setlength{\oddsidemargin}{0in}
\setlength{\textwidth}{6.5in}
\setlength{\textheight}{8.8in}
\setlength{\topmargin}{0in}
\setlength{\headheight}{18pt}



\title{Modélisation et résolution pour l'optimisation}
\author{Noms}

\begin{document}

\maketitle

\vspace{0.5in}


\section{Problèmes d'optimisation sous contraintes}
\subsection{Modélisation du problème }
\begin{enumerate}[(i)]
    \item 
    \begin{flalign*}
        & x_{ij} := \text{ Fréquence de station $i$ à station $j$} && \\
        & t_i := \text{ Région de station $i$} && \\ && \\
        & X = \{x_{11}, \dotsc, x_{nn}\} \cup \{r_1, \dotsc, r_n\} && \\
        & D = \{d_{x_{11}}, \dotsc d_{x_{nn}}\} \cup \{d_{r_1}, \dotsc, d_{r_n}\} \text{ avec pour tout } i, j: d_{x_{ij}} = \mathbb{N}, r_i = \{1, \dotsc, k\} && \\
        & C = \{\forall i \in \{1, \dotsc, n\}: \exists j, k \neq i \Rightarrow \delta_i = |x_{ij} - x_{ji}|\} && \\
        & \cup \{\forall x_{ij} \in X: \exists k \neq i, l \neq j \Rightarrow |x_{ij} - x_{jl}| \geq \Delta_{ij}, |x_{ij} - x_{ki}| \geq \Delta_{ij}\}
    \end{flalign*}
    \begin{itemize}
        \item 
        $ \text{min } |X| $
        \item 
        $ \underset{x_i \in X}{\text{min }} x_i $
        \item 
        $ \text{min }(\underset{x_i \in X}{\text{max }} x_i - \underset{x_i \in X}{\text{min }} x_i) $
    \end{itemize}

\end{enumerate}

\subsection{Instance XCSP3}
Le code COP.py (cf annexe) nous permet de générer une instance XCSP3 à partir de nos données. Ainsi partant du jeu de données data.json suivant : 
\begin{minted}{json}
    {
        "stations": [
                { "num":  0, "region":  0, "delta": 140, "emetteur": [ 14, 28, 42, 56, 70, 84, 98, 112, 126, 140, 154, 168, 182, 196 ], "recepteur": [ 14, 28, 42, 56, 70, 126, 140, 154, 182, 196 ] },
                { "num":  1, "region":  1, "delta": 140, "emetteur": [ 14, 28, 42, 56, 70, 84, 98, 112, 126, 140, 154, 168, 182, 196 ], "recepteur": [ 14, 28, 42, 56, 70, 98, 140, 154, 168, 182, 196 ] },
                { "num":  2, "region":  1, "delta": 140, "emetteur": [ 14, 28, 42, 56, 70, 84, 98, 112, 126, 140, 154, 168, 196 ], "recepteur": [ 14, 28, 42, 56, 70, 84, 98, 112, 126, 140, 154, 168, 182, 196 ] }
        ],
        "regions": [2, 3],
        "interferences": [
                { "x": 0, "y": 1, "Delta": 20 },
                { "x": 0, "y": 2, "Delta": 30 }
        ],
        "liaisons": [
                { "x": 1, "y": 2 }
        ]
    }
\end{minted}

Nous obtenons l'instance XCSP3 suivante : 
\begin{minted}{xml}
    <instance format="XCSP3" type="CSP">
    <variables>
        <array id="fe" note="Variables pour les fréquences d'émission et de réception pour chaque station" size="[3]">
        <domain for="fe[0] fe[1]"> 14 28 42 56 70 84 98 112 126 140 154 168 182 196 </domain>
        <domain for="fe[2]"> 14 28 42 56 70 84 98 112 126 140 154 168 196 </domain>
        </array>
        <array id="fr" size="[3]">
        <domain for="fr[0]"> 14 28 42 56 70 126 140 154 182 196 </domain>
        <domain for="fr[1]"> 14 28 42 56 70 98 140 154 168 182 196 </domain>
        <domain for="fr[2]"> 14 28 42 56 70 84 98 112 126 140 154 168 182 196 </domain>
        </array>
    </variables>
    <constraints>
        <intension> eq(dist(fe[0],fr[0]),140) </intension>
        <intension> eq(dist(fe[1],fr[1]),140) </intension>
        <intension> eq(dist(fe[2],fr[2]),140) </intension>
        <intension> ge(dist(fe[0],fe[1]),20) </intension>
        <intension> ge(dist(fr[0],fr[1]),20) </intension>
        <intension> ge(dist(fe[0],fe[2]),30) </intension>
        <intension> ge(dist(fr[0],fr[2]),30) </intension>
        <nValues>
        <list> fe[0] fr[0] </list>
        <condition> (le,2) </condition>
        </nValues>
        <nValues>
        <list> fe[1] fe[2] fr[1] fr[2] </list>
        <condition> (le,3) </condition>
        </nValues>
    </constraints>
    </instance>
\end{minted}

\subsection{Comparaison des différents solveurs}
Dans cette partie nous allons évaluer les performances des différents solveurs suivants : 
"Choco", "MiniZinc", "Gecode", "CBC" et "Google OR-Tools". Ces évaluations seront basées sur une comparaison du temps d'exécution ainsi
que sur une mesure de performance que nous expliciterons dans la partie dédiée.
\subsubsection{Comparaison en temps}
\subsubsection{Comparaion des performances}


\section{Problèmes de satisfaction de contraintes valuées}
\subsection{Modélisation du problème }
\subsection{Instance WSCP}
\subsection{Résolution avec le solveur Toulbar2}

\end{document}
